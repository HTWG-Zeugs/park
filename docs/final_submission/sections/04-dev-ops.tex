\section{DevOps}

Im Folgenden werden alle Aspekte des DevOps-Prozesses beschrieben, 
die für die Entwicklung und den Betrieb der Anwendung relevant sind. 
Dies beinhaltet die Bereitstellung von Umgebungen, Rollen und Serviceaccounts, 
Pipelines und die Veröffentlichung neuer Features, die Einrichtung neuer Mandanten, 
das Monitoring und das Load Testing.

\subsection{Environments}

Für die Entwicklung und den Betrieb der Anwendung werden verschiedene Umgebungen (Environments) verwendet.
Für den Betrieb der Anwendung wird die \glqq{}Production\grqq{}-Umgebung verwendet, für 
die Entwicklung wird die \glqq{}Staging\grqq{}-Umgebung verwendet. 
Eine weitere \glqq{}Development\grqq{}-Umgebung wurde aufgrund der geringen Anzahl an Entwicklern 
nicht eingerichtet. Werden für die lokale Entwicklung Cloud-Dienste wie Firestore oder 
Storage Buckets benötigt, werden der Einfachheit halber diese aus der \glqq{}Staging\grqq{}-Umgebung 
verwendet.

Jede Umgebung wird in einem eigenen Projekt in der Google Cloud Platform betrieben.
So wird eine klare Trennung zwischen den Umgebungen gewährleistet.
Da beide Umgebungen über IAC (Infrastructure as Code) verwaltet werden,
sind die Umgebungen identisch und können außerdem schnell wiederhergestellt werden.

\subsubsection{Ressourcen pro Umgebung}

Jede Umgebung hat ein eigenes Kubernetes-Cluster, in dem die Anwendung läuft.
Bis auf die Anzahl der Nodes sind die Cluster identisch. 
In der \glqq{}Production\grqq{}-Umgebung sind mehr Nodes vorhanden, um eine höhere Verfügbarkeit zu gewährleisten.


\subsection{Roles and Role Mapping}
\subsection{Pipelines and Release of new Features}
\subsection{New Tenants}
\subsection{Monitoring}
\subsection{Load Testing}