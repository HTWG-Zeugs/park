\section{Requirements}

PARK ist eine Software-as-a-Service (SaaS) Anwendung für die Verwaltung von Parkeinrichtungen für Gemeinden, oder Unternehmen.
Die Anwendung stellt eine umfassende Palette an Funktionen bereit, um den Parkbetrieb zu optimieren, 
die User Experience zu verbessern und die Verwaltung der Parkinfrastruktur effizienter zu gestalten.  \\


\textbf{Hauptziele:}

\begin{itemize}
    \item \textbf{Integration in die Parkinfrastruktur:} Die Anwendung wird in die Parkbereichs- und Garageninfrastruktur integriert, um die automatisierte Ein- und Ausfahrt von Fahrzeugen, sowie die Verifizierung von Zahlungen für Park- und Ladevorgänge zu ermöglichen.
    \item \textbf{Verwaltung der Parkinfrastruktur:} Die Parkinfrastrutktur wird digital abgebildet und kann flexibel an Veränderungen in der realen Welt angepasst werden. Die Integration mit den Prozessen für Parken und Laden sowie mit der Mängelverwaltung ermöglicht umfangreiche Analysen und einen ganzheitlichen Überblick über die Infrastruktur
    \item \textbf{Globale Bereitstellung:} Die SaaS Anwendung ist für globale Skalierbarkeit ausgelegt und ermöglicht eine automatisierte Bereitstellung für Kunden mit unterschiedlichen Bedürfnissen.
\end{itemize}


Funktional ermöglicht die Anwendung die digitale Abbildung der realen Infrastruktur, um die Belegung, Parkflächen, Ladestationen, Mängel und Benutzer an zentraler Stelle zu verwalten. Umfassende Benutzerverwaltung erlaubt verschiedene Sichten auf die Applikation, sodass alle Stakeholder ihren Bedürfnissen entsprechend Zugang zu unterschiedlichen Funktionen erhalten. Automatisierte Bereitstellung ermöglicht es für Tenants mit unterschiedlichen Bedürfnissen, eine auf sie zugeschnittene Instanz der Anwendung erstellen zu können.


\subsection{System Context}
System context diagram und Beschreibung der angrenzenden Systeme


\subsection{Use Case Overview}
Use Case Diagramm und kurze Beschreibung alles Use Cases und Aktoren.

\renewcommand{\arraystretch}{1.5}
{\rowcolors{2}{}{gray!20}
\begin{longtable}{l p{10cm}}
    \textbf{Actor} & \textbf{Beschreibung} \\ [1ex]
    Customer & Der Endkunde, der die Parking Infrastruktur nutzt \\ [0.5ex]
    Tenant Staff & Mitarbeiter (bspw. Financial oder Operations) des Tenants \\ [0.5ex]
    Tenant Administrator & Administrator auf Tenant-Ebene \\ [0.5ex]
    Solution Administrator & Administrator, der die Lösung bereitstellt \\ [0.5ex]
    Lead & Potenzielle Kunden, die eine Instanz der Lösung erstellen möchten \\ 
\end{longtable}}