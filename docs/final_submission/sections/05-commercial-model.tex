\section{Commercial Model}
\subsection{Tenant Types}

\paragraph{Free Tenant}
Der Free-Plan richtet sich an Nutzer, die grundlegende Funktionen zur Verwaltung von Standorten und Infrastruktur benötigen. Die zentralen Merkmale dieses Plans sind:
\begin{itemize}
	\item \textbf{Parkhausverwaltung:} Erstellung und Verwaltung mehrerer Parkhäuser zur effizienten Organisation von Standorten.
	\item \textbf{Defektmanagement:} Nachverfolgung und Bearbeitung von Defekten zur schnellen Problemlösung.
	\item \textbf{Integration von E-Mobility-Ladestationen:} Einbindung und Überwachung der Ladeinfrastruktur für Elektrofahrzeuge.
\end{itemize}
Der Free-Plan stellt essenzielle Werkzeuge zur Verwaltung von Parkhäusern und Ladestationen bereit, ohne monatliche Kosten zu verursachen.

\paragraph{Premium Tenant}
Der Premium-Plan bietet die gleiche Grundfunktionalität wie der Free-Plan, richtet sich jedoch an Organisationen mit höheren Anforderungen an Benutzerverwaltung, Analysen und Systemleistung.
Zu den Funktionen gehören:
\begin{itemize}
	\item \textbf{Benutzerverwaltung:} Unbegrenzte Benutzeranzahl mit erweiterten Zugriffskontrollen.
	\item \textbf{E-Mobility-Ladestationen:} Erweiterte Integration und Verwaltung von Ladeinfrastrukturen.
	\item \textbf{Mehrsprachigkeit:} Unterstützung mehrerer Sprachen für internationale Nutzung.
	\item \textbf{Analytik-Dashboard:} Bereitstellung detaillierter Analysen zur Prozessoptimierung.
	\item \textbf{Optimierte Performance:} Schnellere Ladezeiten und höhere Systemzuverlässigkeit.
	\item \textbf{Premium-Support:} Priorisierte Kundenbetreuung für effiziente Problemlösungen.
	\item \textbf{Erhöhte Sicherheit:} Erweiterte Sicherheitsfunktionen zum Schutz sensibler Daten.
\end{itemize}
\textbf{Kosten:}
\begin{itemize}
	\item \textbf{79,90 €/Monat} Grundgebühr.
	\item \textbf{1 €/Benutzer.}
	\item \textbf{5 €/10.000 Backend-Requests.}
\end{itemize}

\paragraph{Enterprise Tenant}
Der Enterprise-Plan umfasst die Funktionen des Premium-Plans und ist für große Organisationen konzipiert, die maximale Leistung, Sicherheit und Support benötigen.
Er beinhaltet:
\begin{itemize}
	\item \textbf{Anpassbarkeit:} Bereitstellung einer dedizierten Domain.
	\item \textbf{Maximale Performance:} Höchste Geschwindigkeit und Systemstabilität.
	\item \textbf{Priorisierter Support:} Bevorzugte Bearbeitung aller Anfragen.
	\item \textbf{Erweiterte Sicherheitsstandards:} Umfassender Schutz sensibler Daten.
\end{itemize}
\textbf{Kosten:}
\begin{itemize}
	\item \textbf{129,90 €/Monat} Grundgebühr.
	\item \textbf{5 €/10.000 Backend-Requests.}
\end{itemize}

\subsection{Pricing Model}
Das Preismodell der Plattform bietet flexible Tarife, die auf verschiedene Nutzeranforderungen zugeschnitten sind.
Die Preisstruktur ermöglicht eine skalierbare Nutzung entsprechend der Anzahl der Benutzer und Backend-Anfragen.

\begin{table}[h!]
	\centering
	\caption{Preisübersicht der Tenant-Typen}
	{\rowcolors{2}{}{gray!20}
		\begin{tabularx}{\textwidth}{|l|X|X|X|}
			\hline
			\textbf{Plan} & \textbf{Grundgebühr (€/Monat)} & \textbf{Preis pro Benutzer (€/Monat)} & \textbf{Preis pro 10.000 Backend-Requests (€/Monat)} \\ \hline
			Free          & 0,00                           & -                                     & -                                                    \\ \hline
			Premium       & 79,90                          & 1,00                                  & 5,00                                                 \\ \hline
			Enterprise    & 129,90                         & -                                     & 5,00                                                 \\ \hline
		\end{tabularx}}
	\label{tab:pricing}
\end{table}

Diese Preisgestaltung ermöglicht sowohl kleinen als auch großen Organisationen eine anpassbare und kosteneffiziente Nutzung der Plattform.

\subsection{Cost Model}
Das Kostenmodell basiert auf einer optimierten Ressourcennutzung und unterscheidet sich je nach Tenant-Typ:
\begin{itemize}
	\item \textbf{Free-Plan:} Mandanten teilen sich einen Kubernetes-Namespace, eine gemeinsame Firestore-Datenbank und gemeinsamen Cloud-Storage.
	\item \textbf{Premium-Plan:} Gemeinsame Nutzung eines Kubernetes-Namespaces mit erhöhter Ressourcenallokation, gemeinsamer Firestore-Datenbank und Cloud-Storage.
	\item \textbf{Enterprise-Plan:} Dedizierte Kubernetes-Namespaces, eigene Firestore-Datenbank und eigener Cloud-Storage pro Mandant.
	\item \textbf{Geteilte Services:} Drei zentrale Cloud-Run-Services werden von allen Mandanten genutzt.
\end{itemize}

Für unsere Kalkulation haben wir den Google Cloud Pricing Calculator verwendet, um die Kosten für die verschiedenen Ressourcen zu schätzen. Hier der Link zu unserer Konfiguration \href{https://cloud.google.com/products/calculator?hl=de&dl=CjhDaVExWXpBNE5UUm1OaTA0TUdKaUxUUTFPVFl0T0dGa09DMWxZVEEzTVdZeVlXSm1ZalVRQVE9PRAJGiQzREI3QkJCQS04Nzc0LTQ5RTEtODY5Qy1DQjVCREI5NzlGMDM}{Calculator}. Hier sind jedoch nur die Kosten für eine Umgebung zu sehen, da wir eine identische Production und Staging Umgebung haben, verdoppeln sich die Kosten.  \\

\begin{table}[h!]
	\centering
	\caption{Ressourcennutzung und monatliche Kostenschätzung}
	{\rowcolors{2}{}{gray!20}
		\begin{tabularx}{\textwidth}{|l|X|X|X|}
			\hline
			\textbf{Ressource}    & \textbf{Free-Plan}              & \textbf{Premium-Plan} & \textbf{Enterprise-Plan (pro Mandant)} \\ \hline
			Kubernetes Namespace  & 135 €/Monat                     & 135 €/Monat           & 400 €/Monat (gesamt) = 40 €/Monat (pro Tenant)                          \\ \hline
			Firestore-Datenbank   & 2 €/Monat                       & 2 €/Monat             & 2 €/Monat                              \\ \hline
			Cloud Storage         & 4 €/Monat                       & 4 €/Monat             & 4 €/Monat                              \\ \hline
			Cloud Functions       & 6 €/Monat                       & 6 €/Monat             & 6 €/Monat                              \\ \hline
			Geteilte Services     & \multicolumn{3}{c|}{60 €/Monat}                                                                  \\ \hline
			\textbf{Gesamtkosten} & 167 €/Monat                     & 167 €/Monat           & $\sim 54$ €/Monat                      \\ \hline
		\end{tabularx}}
	\label{tab:costmodel}
\end{table}

In Tabelle \ref{tab:costmodel} werden die geschätzten Kosten für die verschiedenen Pläne dargestellt.
Diese Kosten sind auf Staging und Production Umgebung zusammen berechnet.
Dabei wurden die Enterprise Namespace Kosten durch zehn geteilt (Annahme wir haben 10 Enterprise Kunden) und die der Geteilte Services Anteil von 20€ auch nochmal , um die Verteilung der Ausgaben für einzelne Mandanten zu berücksichtigen.
Unter geteilten Kosten sind die Kosten für die drei Services (Authentication Service mit Datenbank, Infrastructure Management Service mit Datenbank und Sign Up Frontend), die von allen Mandanten gemeinsam genutzt werden, zusammengefasst.
\\
Die wichtigsten Punkte sind:
\begin{itemize}
	\item \textbf{Free Plan:} Die Gesamtkosten belaufen sich auf geschätzte 167 €/Monat durch die gemeinsame Nutzung von Ressourcen.
	\item \textbf{Premium Plan:} Die Gesamtkosten belaufen sich auf geschätzte 167 €/Monat durch die gemeinsame Nutzung von Ressourcen.
	\item \textbf{Enterprise Plan:} Jeder Enterprise-Mandant verursacht individuelle Kosten von rund 54 €/Monat. Dies ist bedingt durch die exklusive Zuweisung von Ressourcen wie Kubernetes-Namespace, Firestore-Datenbank und Cloud Storage.
\end{itemize}

Dieses Kostenmodell bietet eine klare Grundlage für die wirtschaftliche Planung und Skalierung der Plattform.

\paragraph{Worst-Case-Szenario}
Wir haben nur Free Kunden, die alle die maximale Anzahl an Backend-Requests verursachen und keine Premium und Enterprise Kunden.
Das heißt wir Kubernetes Cluster läuft ohne dass die Ressourcen genutzt werden und die geteilten Services bleiben auch gleich teuer.
Dann fallen folgende Kosten an:
\begin{itemize}
	\item \textbf{Kubernetes Namespace:} 135 (free) + 135 (premium) + 400 (enterprise) = 670 €/Monat
	\item \textbf{Firestore-Datenbank:} 2 €/Monat
	\item \textbf{Cloud Storage:} 4 €/Monat
	\item \textbf{Cloud Functions:} 6 €/Monat
	\item \textbf{Geteilte Services:} 60 €/Monat
	\item \textbf{Gesamtkosten:} 742 €/Monat
\end{itemize}
Das heißt wir würden im Worst Case Szenario 742 €/Monat minus machen.

\paragraph{Realitisches Szenario}
Über 10 Enterprise Kunden, die alle auch die maximale Anzahl an Backend-Requests verursachen (10.000 für einfachere Rechnung) und über 10 Premium Kunden die auch ebenfalls die maximale Backend-Requests (10.000 für einfachere Rechnung) erreichen und zusammen 40 Nutzer haben.
Free Kunden und Premium Kunden sind irrelevant da sie monatlich immer die gleichen Fixkosten verursachen.
Dann fallen folgende Kosten an:
\begin{itemize}
	\item \textbf{Kubernetes Namespace:} 135 (free) + 135 (premium) + 400 (enterprise) = 670 €
	\item \textbf{Firestore:} 2 (free) + 2 (premium) + 20 (enterprise) = 24 €
	\item \textbf{Cloud Storage:} 4 (free) + 4 (premium) + 40 (enterprise) = 48 €
	\item \textbf{Cloud Functions:} 6 (free) + 6 (premium) + 60 (enterprise) = 72 €
	\item \textbf{Geteilte Services:} 60 €
	\item \textbf{Gesamtkosten:} 874 €/Monat
\end{itemize}
Das heißt wir würden im Realistischen Szenario 874 €/Monat an Kosten zahlen.
Jedoch folgende Einnahmen generieren:
\begin{itemize}
	\item \textbf{Enterprise Kunden:} 10 * 129,90 € = 1299 €
	\item \textbf{Premium Kunden:} 10 * 79,90 € = 799 €
	\item \textbf{Benutzer:} 40 * 1,00 € = 40 €
	\item \textbf{Backend-Requests:} 20 * 5,00 € = 50 €
	\item \textbf{Gesamteinnahmen:} 2238 €
\end{itemize}

Das heißt wir würden im Realistischen Szenario (Annahme nur 10 Enterprise und 10 Premium Tenants ist eher pessimistisch) 1364 €/Monat an Gewinn machen.

\paragraph{Best-Case-Szenario}
Das Best-Case-Szenario wäre, wenn wir sehr viele Enterprise Kunden haben, die alle weit über 10.000 Backend-Requests senden und viele Premium Kunden, die viele Nutzer erstellt haben und monatlich viele (über 10.000 pro Tenant) Backend-Requests senden.
Die Free Kunden sind irrelevant da sie monatlich immer die gleichen Fixkosten verursachen und uns keinen Gewinn einbringen.
