\section{Commercial Model}
\subsection{Tenant Types}

\paragraph{Free Tenant}
Der Free-Plan richtet sich an Nutzer, die grundlegende Funktionen zur Verwaltung von Standorten und Infrastruktur benötigen. Die zentralen Merkmale dieses Plans sind:
\begin{itemize}
	\item \textbf{Parkhausverwaltung:} Erstellung und Verwaltung mehrerer Parkhäuser zur effizienten Organisation von Standorten.
	\item \textbf{Defektmanagement:} Nachverfolgung und Bearbeitung von Defekten zur schnellen Problemlösung.
	\item \textbf{Integration von E-Mobility-Ladestationen:} Einbindung und Überwachung der Ladeinfrastruktur für Elektrofahrzeuge.
\end{itemize}
Der Free-Plan stellt essenzielle Werkzeuge zur Verwaltung von Parkhäusern und Ladestationen bereit, ohne monatliche Kosten zu verursachen.

\paragraph{Premium Tenant}
Der Premium-Plan bietet die gleiche Grundfunktionalität wie der Free-Plan, richtet sich jedoch an Organisationen mit höheren Anforderungen an Benutzerverwaltung, Analysen und Systemleistung.
Zu den Funktionen gehören:
\begin{itemize}
	\item \textbf{Benutzerverwaltung:} Unbegrenzte Benutzeranzahl mit erweiterten Zugriffskontrollen.
	\item \textbf{E-Mobility-Ladestationen:} Erweiterte Integration und Verwaltung von Ladeinfrastrukturen.
	\item \textbf{Mehrsprachigkeit:} Unterstützung mehrerer Sprachen für internationale Nutzung.
	\item \textbf{Analytik-Dashboard:} Bereitstellung detaillierter Analysen zur Prozessoptimierung.
	\item \textbf{Optimierte Performance:} Schnellere Ladezeiten und höhere Systemzuverlässigkeit.
	\item \textbf{Premium-Support:} Priorisierte Kundenbetreuung für effiziente Problemlösungen.
	\item \textbf{Erhöhte Sicherheit:} Erweiterte Sicherheitsfunktionen zum Schutz sensibler Daten.
\end{itemize}
\textbf{Kosten:}
\begin{itemize}
	\item \textbf{79,90 €/Monat} Grundgebühr.
	\item \textbf{1 €/Benutzer.}
	\item \textbf{5 €/10.000 Backend-Requests.}
\end{itemize}

\paragraph{Enterprise Tenant}
Der Enterprise-Plan umfasst die Funktionen des Premium-Plans und ist für große Organisationen konzipiert, die maximale Leistung, Sicherheit und Support benötigen.
Er beinhaltet:
\begin{itemize}
	\item \textbf{Anpassbarkeit:} Bereitstellung einer dedizierten Domain.
	\item \textbf{Maximale Performance:} Höchste Geschwindigkeit und Systemstabilität.
	\item \textbf{Priorisierter Support:} Bevorzugte Bearbeitung aller Anfragen.
	\item \textbf{Erweiterte Sicherheitsstandards:} Umfassender Schutz sensibler Daten.
\end{itemize}
\textbf{Kosten:}
\begin{itemize}
	\item \textbf{129,90 €/Monat} Grundgebühr.
	\item \textbf{5 €/10.000 Backend-Requests.}
\end{itemize}

\subsection{Pricing Model}
Das Preismodell der Plattform bietet flexible Tarife, die auf verschiedene Nutzeranforderungen zugeschnitten sind.
Die Preisstruktur ermöglicht eine skalierbare Nutzung entsprechend der Anzahl der Benutzer und Backend-Anfragen.

\begin{table}[h!]
	\centering
	\caption{Preisübersicht der Tenant-Typen}
	{\rowcolors{2}{}{gray!20}
		\begin{tabularx}{\textwidth}{|l|X|X|X|}
			\hline
			\textbf{Plan} & \textbf{Grundgebühr (€/Monat)} & \textbf{Preis pro Benutzer (€/Monat)} & \textbf{Preis pro 10.000 Backend-Requests (€/Monat)} \\ \hline
			Free          & 0,00                           & -                                     & -                                                    \\ \hline
			Premium       & 79,90                          & 1,00                                  & 5,00                                                 \\ \hline
			Enterprise    & 129,90                         & -                                     & 5,00                                                 \\ \hline
		\end{tabularx}}
	\label{tab:pricing}
\end{table}

Diese Preisgestaltung ermöglicht sowohl kleinen als auch großen Organisationen eine anpassbare und kosteneffiziente Nutzung der Plattform.

\subsection{Cost Model}
Das Kostenmodell basiert auf einer optimierten Ressourcennutzung und unterscheidet sich je nach Tenant-Typ:
\begin{itemize}
	\item \textbf{Free-Plan:} Mandanten teilen sich einen Kubernetes-Namespace, eine gemeinsame Firestore-Datenbank und gemeinsamen Cloud-Storage.
	\item \textbf{Premium-Plan:} Gemeinsame Nutzung eines Kubernetes-Namespaces mit erhöhter Ressourcenallokation, gemeinsamer Firestore-Datenbank und Cloud-Storage.
	\item \textbf{Enterprise-Plan:} Dedizierte Kubernetes-Namespaces, eigene Firestore-Datenbank und eigener Cloud-Storage pro Mandant.
	\item \textbf{Geteilte Services:} Drei zentrale Cloud-Run-Services werden von allen Mandanten genutzt.
\end{itemize}

\begin{table}[h!]
	\centering
	\caption{Ressourcennutzung und monatliche Kostenschätzung}
	{\rowcolors{2}{}{gray!20}
		\begin{tabularx}{\textwidth}{|l|X|X|X|}
			\hline
			\textbf{Ressource}    & \textbf{Free-Plan}              & \textbf{Premium-Plan} & \textbf{Enterprise-Plan (pro Mandant)} \\ \hline
			Kubernetes Namespace  & 135 €/Monat                     & 135 €/Monat           & 400 €/Monat                            \\ \hline
			Firestore-Datenbank   & 2 €/Monat                       & 2 €/Monat             & 2 €/Monat                              \\ \hline
			Cloud Storage         & 4 €/Monat                       & 4 €/Monat             & 4 €/Monat                              \\ \hline
			Cloud Functions       & 6 €/Monat                       & 6 €/Monat             & 6 €/Monat                              \\ \hline
			Geteilte Services     & \multicolumn{3}{c|}{60 €/Monat}                                                                  \\ \hline
			\textbf{Gesamtkosten} & 167 €/Monat                     & 167 €/Monat           & $\sim 54$ €/Monat                      \\ \hline
		\end{tabularx}}
	\label{tab:costmodel}
\end{table}
